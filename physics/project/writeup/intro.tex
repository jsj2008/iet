\section{Introduction}
Simulating fluids is an important problem in computer graphics. 
Physical models of fluids allow for very convincing simulations to be created.
It was thought that simulating the physical models of fluids was too complex for real time applications.
The solver presented by Stam~\cite{DBLP:conf/siggraph/Stam99a} however allows for the stable physical simulation of fluids in real time.
The stability of the model is integral to its ability to be simulated in real-time, as previous techniques typically used unstable methods to solve the equations governing a fluid.
This stability allows the user to take larger time steps and achieve faster simulations without compromising the stability of the system.

The solver presented uses the \emph{Navier-Stokes} equations to simulate fluid flow.
Fluid solvers in computer graphics typically weight real-time interactivity higher than strict physical accuracy, which engineering-type applications would be interested in.
Most computer graphics applications tended to use simple primitives such as particles~\cite{Reeves83particlesystems}.
The complexity of such simulations was increased with the introduction of random turbulences~\cite{Chen97real-timefluid}.
These exhibit rotational motion and are mass conserving. 
However flows built up from a superposition of flow primitives will not react dynamically to forces applied by the user in real-time as we desire.

Models which use the \emph{Navier-Stokes} equations first were implemented in 2-dimensions.
Gamito et al. used a vortex method with a poisson solver to create two-dimensional animations of fluids~\cite{Gamito95two-dimensionalsimulation}.
Chen et al. later animated water surfacts from the pressure term given by the \emph{Navier-Stokes} equations~\cite{Chen97real-timefluid}.
These methods of solving Navier-Stokes is both unstable and limited to two dimensions. 
Many effects arise from using the full \emph{Navier-Stokes} equations such as swirling motion and flows past objects.
Explicit solvers for the \emph{Navier-Stokes} equations were created by Foster and Metaxas~\cite{Foster97modelingthe} but have the disadvantage of being unstable for large time steps.
This instability sets limits on the speed and interactivity of these simulations, where a user might have to restart the simulation if it ``blows up'' unexpectedly.

The method created by Stam is both stable and uses the three dimensional form of the \emph{Navier-Stokes} equations.
It is able to use timesteps which are much larger than that of Foster and Metaxas and thus is able to achieve real-time speed.
The method uses Lagrangian and implicit methods to solve the \emph{Navier-Stokes} equations over the Eulerian techniques previously used.
The model presented is not accurate enough for engineering applications as it suffers from ``numerical dissipation'' but it is ideal for real time applications where the user is repeatedly applying new forces.
